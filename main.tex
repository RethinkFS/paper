% Copyright (c) 2022 Hans Wan, Chiro Liang and ReiSakuma
% All rights reserved.
% This paper is publicised for non-commercial learning purpose only.
% THIS PAPER IS PROVIDED ON AN "AS IS" BASIS, WITHOUT WARRANTIES OF ANY KIND,
% EITHER EXPRESS OR IMPLIED, INCLUDING BUT NOT LIMITED TO NON-INFRINGEMENT,
% MERCHANTABILITY OR FIT FOR A PARTICULAR PURPOSE.

\documentclass[a4paper]{article}
\usepackage[svgnames]{xcolor}
\usepackage{hcr-cumcm}
\usepackage[style=gb7714-2015,backend=biber]{biblatex}
\addbibresource{bibdata.bib}

\newcommand{\subsubsubsection}[1]{\paragraph{#1}\mbox{}\\}
\setcounter{secnumdepth}{4} % how many sectioning levels to assign numbers to
\setcounter{tocdepth}{4} % how many sectioning levels to show in ToC

\usepackage{algorithm}
\usepackage{algpseudocode}
\usepackage{amsmath}
\renewcommand{\algorithmicrequire}{\textbf{Input:}}  % Use Input in the format of Algorithm
\renewcommand{\algorithmicensure}{\textbf{Output:}} % Use Output in the format of Algorithm

\lstset{frame=tb,
  language=C++,
  aboveskip=3mm,
  belowskip=3mm,
  showstringspaces=false,
  columns=flexible,
  basicstyle={\small\ttfamily},
%   numbers=none,
  numberstyle=\tiny\color{gray},
  keywordstyle=\color{blue},
  commentstyle=\color{dkgreen},
  stringstyle=\color{mauve},
  breaklines=true,
  breakatwhitespace=true,
  tabsize=2
}

\title{AquaFS 开发设计文档}

\begin{document}
    \pagestyle{plain}
    \maketitle
    \newpage
    %% ABSTRACT
    \section*{摘要}
\addcontentsline{toc}{section}{摘要}
    \newpage
    \tableofcontents
    \newpage
    %% CONTENT
    \section{概述}

本项目 AquaFS 是一个智能的 Flash 友好的文件系统,旨在为闪存设备提供更好的性能和更长的寿命。
本文档将从多个角度介绍 AquaFS 的设计与实现。

AquaFS 与传统文件系统相比,有以下几个特点:

\begin{itemize}
    \item 与 Ext4 等原地更新的文件系统不同,AquaFS 采用了日志结构,以减少写放大。
    \item 与传统软件文件系统不同,AquaFS 与硬件结合共同设计,降低硬件成本的同时提升性能。
    \item 与普通的单一文件系统不同,AquaFS 采用了多系统模块结合的方法,以提升系统的灵活性。
    \item 设计了专门的参数调整逻辑,以动态适应不同的工作负载。
    \item 设计了智能的读写逻辑,不同的工作负载能够由不同的模块处理或共同处理,以提升系统的性能。
    \item 优化垃圾回收逻辑,硬件上取消垃圾回收,软件上可调整预留空间与垃圾回收频率,以均衡调整性能与寿命。
    \item 其他特性。
\end{itemize}

后文将结合背景调研、需求分析、系统设计、系统实现、系统测试等方面,对 AquaFS 的各种设计特性进行详细介绍。

    \section{需求分析与调研}

\subsection{需求分析}

根据题目描述,我们需要实现一个“更加智能的 Flash 文件系统”。

Flash介质因本身的擦除特性,给上层文件系统带来与普通磁盘、内存文件系统不同的数据管理模式。
同时,Flash的写入放大、寿命以及后期稳定性下降等问题也给文件系统的设计带来的一定的挑战。
传统的Flash文件系统并没有很好地解决这些稳定性相关问题。
这里希望寻找一种更智能、更合适的Flash文件系统设计,来更好地平衡Flash的性能与稳定性。
可以思考的方向包括但不限于:数据压缩算法(可以是自适应的压缩算法),检错、纠错、纠删码(可以是联合信源信道编码),数据块选择、擦除策略,Cache机制。

结合赛题内容,我们需要完成的文件系统需要具备以下特性:

\begin{itemize}
  \item {\bf{适合 Flash 介质}}:文件系统需要能够适应并缓解 Flash 的写入放大、寿命、后期稳定性等诸多 Flash 存储介质特有的问题
  \item {\bf{更智能}}:文件系统需要使用多种策略,在软件算法层面平衡性能与寿命
\end{itemize}

\subsubsection{背景调研}

\subsubsubsection{往年实现分析}

本题在去年已经有队伍完成,他们的仓库在 \url{https://gitlab.eduxiji.net/why/project788067-124640}。
在选题之时,我们对其进行了调研。

他们队伍完成了一个基于 UBIFS 的适用于裸 Flash 设备的 YOUBIFS 文件系统,从以下几个方面对其进行了优化:

\begin{itemize}
  \item {\bf{数据压缩模块}}:使用了预压缩和自适应压缩结合的方法,均衡压缩比和写入速度
  \item {\bf{纠错编码模块}}:实现自适应生命周期、CRC+RAID5两种纠错方案
  \item {\bf{Cache 机制}}:添加读写缓冲
  \item {\bf{冷热数据识别}}:使用了冷热识别的纠错以提高 I/O 速度
\end{itemize}

\begin{figure}[htbp]
  \centering
  \includegraphics[width=0.7\textwidth]{fig/YOUBIFS项目框图.png}
  \caption{YOUBIFS 系统架构}
  \label{youbifs}
\end{figure}

YOUBIFS 的系统架构如图 \ref{youbifs} 所示。
看起来 YOUBIFS 的实现已经非常满足题目的需求,能够很好地回答题目中提出的几个问题。
但是我们发现,YOUBIFS 的系统实现也存在一些不足之处:

\begin{itemize}
  \item {\bf{Nor Flash 应用场景}}
    
  YOUBIFS 的实现中,文件系统的实现是针对于 Nor Flash 的。
  Nor Flash 在实现上与 Nand Flash 有很大的不同,因此 YOUBIFS 的实现并不能直接应用于 Nand Flash 上,应用场景受到了限制。
  
  同时,Nor Flash 一般用于嵌入式设备,而 Nand Flash 一般用于移动设备或高性能设备,因此 YOUBIFS 的实现也不能直接应用于移动设备或固态硬盘上。
  YOUBIFS 针对于 Nor Flash 的多种写入情况做了优化,但是由于 Nor Flash 的寿命、速度、稳定性等特性,其并不适合在写入负载较高的情况下工作。

  Nor Flash 的寿命一般在 10 万次左右,而 Nand Flash 的寿命一般在 100 万次左右,因此 Nor Flash 广泛用于 BIOS 存储、固件存储等写入负载较低的场景。
  在这些应用场景下,由于存储的数据都是非常重要的系统关键数据,如 Bootloader、系统固件等,因此对于数据的稳定性要求非常高。
  如果系统软件直接在 Nor Flash 上频繁读写,很可能会导致 Nor Flash 的寿命过早耗尽,从而导致系统无法正常启动,造成非常严重的后果。

  因此,YOUFIBS 的实现并不能直接应用于移动设备或固态硬盘上,也不能直接应用于写入负载较高的场景。
  这与 YOUBIFS 的各种优化策略相矛盾,因此我们需要重新设计一个适用于 Nand Flash 的文件系统。

  \item {\bf{“智能”,但是还不够智能}}

  YOUBIFS 的实现中,使用了多种策略来平衡性能与寿命,但是这些策略都是相对固定的,只是能够根据不同的情况进行策略切换。
  其中能够体现“智能”的实现有:

  \begin{enumerate}
    \item 通过判断文件名的方式来判断使用的压缩算法和参数,即压缩算法和等级的自适应
    \item 通过检查压缩效果是否合适来判断是否继续压缩,压缩效果不好可能其数据本身就是压缩文件,则再次不压缩
    \item 利用文件系统中的局部性,判别冷热数据文件
    \item 在 Flash 后期容易产生错误的时候转用纠错能力更强的算法
  \end{enumerate}

\end{itemize}
    \section{系统设计}

AquaFS 是一个模块化的文件系统,以下是 AquaFS 的整体架构图:

\subsection{AquaFS 整体架构}

\begin{lstlisting}
  +---------+ +-------+
  | RocksDB | |  App  |
  +--------++ ++------+
    FS/POSIX|   |VFS/FUSE
+-AquaFS----+---+----------+------+
|           |   |          |Turner|
|     +-----+---+--------+ +------+-----+
|     |   Data Router    | |Configurator|
|     +--+--------------++ +------+-----+
|    SST |              | Data    |
| +------+---+ inode +--+-------+ |
| | AquaZFS  |<------+  ExtFS   | |
| +--+-------+       +---+------+ |
|    |RAID         Extent|        |
|    |   +---------------+Data    |
| +--+---+---------------+------+ |
| |  |   |Zones Allocator|      | |
| +--v---v---------------v------+ |
| |       io_uring/xNVME        | |
| +--------------+--------------+ |
| | Seq Zones    | Conv Zones   | |
+-+--------------+--------------+-+
\end{lstlisting}

以上架构图中的模块简略说明如下:

\begin{enumerate}
    \item App:文件系统请求负载
    \item RocksDB:数据库请求
    \item Data Router:FileSystem 请求路由器,需要判断当前请求是否适合 WAL 优化
    \item Turner:动态调整运行过程中的参数
    \item Configurator:静态调整文件系统参数,建立文件系统时给出建议参数
    \item AquaZFS:经过修改和优化的 ZenFS,支持 RAID 等功能
    \item ExtFS:有 inode 系统的运行于 Seq/Conv Zones 上的通用文件系统
        \begin{enumerate}
            \item 对普通请求,直接使用 Conv Zones
            \item 对 AquaZFS 的新文件,提供 inode 索引等读写优化
            \item 对较大的冷数据文件,分配到 Seq Zones
            \item 将一些可以异地更新的数据以 AquaFS::Extent 形式写入 AquaZFS
        \end{enumerate}
    \item Zones Allocator:为 AquaZFS、ExtFS 提供 Zone 分配服务
    \item Zones:
        \begin{enumerate}
            \item Seq Zones:只能顺序写的 Zones
            \item Conv Zones:可以随机写的 Zones
        \end{enumerate}
    \item AquaFS:整体文件系统
\end{enumerate}

\subsection{模块设计细节}

\subsubsection*{RocksDB 使用 AquaFS}

RocksDB 使用 AquaFS,可以走两种数据通路:FileSystem 和 POSIX 接口。

\textbf{RocksDB 使用 FileSystem 接口使用 AquaFS}

将 AquaFS 编译为 RocksDB 插件,Data Router 使用 FileSystem 接口。

Data Router 主要转发 SST 请求到 AquaZFS,其他请求转发到 ExtFS,并对特殊情况做二者的负载均衡。

\textbf{RocksDB 使用 POSIX 接口使用 AquaFS}

AquaFS 的 Data Router 通过 Kernel Module 或 FUSE 提供 POSIX 访问接口,并智能判断数据负载位置。

\subsubsection*{App 使用 AquaFS}

由于假定 App 并未实现 FileSystem 接口,所以 App 数据可以通过 Kernel Module / FUSE 方式经过 Data Router 到下层。

\subsubsection*{调参模块}

AquaFS 中调参模块主要有两个部分:Configurator、Turner。

Configurator 在文件系统创建前评估当前系统更适合的固定参数,并结合需求给出合适的参数选择和预估的性能区间。

Turner 在文件系统使用过程中保持运行,根据系统当前状态动态调整可改变的参数,以获得更加灵活良好的整体表现。

\textbf{可调参数}

\begin{enumerate}
    \item 固定参数
    \begin{enumerate}
        \item 块大小
        \item 固定 RAID 参数
        \item 数据后端类型
    \end{enumerate}
    \item GC
    \begin{enumerate}
        \item GC 容量阈值
        \item GC 间隔时间
    \end{enumerate}
    \item 动态 RAID
    \begin{enumerate}
        \item 分配时间(GC)
        \item 分配参数(0/1/5...)
    \end{enumerate}
    \item 文件请求分类
    \begin{enumerate}
        \item 分类为 SST、普通数据
        \item 分类冷热文件/数据
    \end{enumerate}
    \item IO 加速方式:io\_uring/xNVME
\end{enumerate}

\subsubsection*{AquaZFS 和 ExtFS}

ExtFS 是主要运行于 Conv Zones 上的针对 ZNS 优化的文件系统。主要特性:

\begin{enumerate}
    \item 必须原地更新的数据放在 Conv Zones 内,如 Superblock 等(?)
    \item 适合异地更新的数据通过 AquaZFS 保存在 Seq Zones 内,如 MetaData(?)
    \item 如果智能检测到 AquaZFS 内部分数据不适合 Seq Zones 存储,则转发到 ExtFS 内处理
    \item 用冗余的 inode 等为 AquaZFS 提供索引,可以动态降低其内存消耗
\end{enumerate}

AquaZFS 是基于 ZenFS 的优化修改,支持以上特性,在保存高性能的同时提升文件系统的灵活性。

\subsubsection*{RAID}

在 AquaZFS 从写盘前到实际写盘之间,存在一层 RAID 逻辑。

在原版 ZenFS 的实现中并没有实现 RAID 的逻辑,其只能保证存储的“记录”的数据正确性,而无法保证“文件”的数据正确性,
也不能在遇到磁盘故障的时候自动处理修复数据。

AquaZFS 在 ZenFS 的基础上实现了 RAID 逻辑,可以在磁盘故障时自动修复数据,且能够根据配置自动使用不同的 RAID 策略。

除了保证数据的正确性,AquaZFS 还可以根据不同的 RAID 策略提供不同的性能。
当前实现的 RAID0 可以以 N 倍加速单线程数据的读写,RAID1 在提供 N 倍读性能的同时,还可以用简单的冗余策略保证数据的安全性。

AquaZFS 的 RAID 有两种基本模式:全盘模式和智能动态分区模式。

在全盘模式下,AquaZFS 会以 SSD 设备为单位应用同一种 RAID 策略,其策略配置写入超级块中,
可以快速将已经在使用的 AquaZFS 转换为 全盘 RAID 格式,并使用超级块在 Meta Zones 中的追加写入逻辑保证配置的正确性。

在智能动态分区模式下,AquaZFS 以 Zones 为单位配置 RAID 策略,可以在不同的 Zones 中使用不同的 RAID 策略。
RAID 策略信息以记录形式写入 Meta Zones 中,并使用 WAL 配合 Snapshot 保证数据的正确性。

在实现的过程中,我们为其他的 RAID 逻辑做了预留,能够快速地实现其他 RAID 逻辑。

\begin{enumerate}
    \item 可灵活配置为:静态固定参数 RAID、智能动态分区 RAID
    \item 可以在用户态驱动 NVME,或者内核态使用 liburing 进行 IO 加速,充分利用多盘优势提升性能
    \item 利用 Turner 提供的建议,在 AquaZFS 垃圾回收时或合并 Extent 时调整 RAID 逻辑,使文件系统在安全性、性能上有更好的权衡点
\end{enumerate}

\subsubsection*{Zones Allocator}

为 AquaZFS 和 ExtFS 提供统一的 Zones 分配服务。

\begin{enumerate}
    \item 让整盘空间得到更加充分的利用,减少由于分开两种子系统造成的空间碎片
    \item 根据历史数据,测算不同 Zones 的寿命和速度,来控制 Zones 的分配逻辑,延长磁盘寿命,提高磁盘吞吐
\end{enumerate}

\subsubsection*{IO 加速}

在 AquaFS 向上提供 FileSystem 接口时,由于负载程序对 FileSystem 接口做了适配,所以可以让负载程序和整个 AquaFS 都跑在用户态。

当 AquaFS 整个运行在用户态,可以使用 xNVME 用户态 NVME 协议驱动,降低内核态用户态切换的性能损失,同时也可用 io\_uring 加速。

若 AquaFS 使用 POSIX 接口,可以使用 VFS 或者 FUSE 接口,此时也可以用 xNVME 或者 io\_uring 进行 IO 加速。

\subsubsection*{智能化}

这个架构的「智能」体现在哪?

\begin{enumerate}
    \item 相比与 ZenFS,灵活性更强
    \begin{enumerate}
        \item 适配没有针对优化的工作负载,智能识别适合 WAL 的数据,用更合适的方式处理
        \item 可调整的静态、动态参数更多
        \item 提供 RAID 功能,并可以动态分配
    \end{enumerate}
    \item 数据安全性更强:RAID 功能
    \item 智能分配请求
    \begin{enumerate}
        \item 在 Data Router 层合理分配 SST、普通数据请求
        \item 在当前请求不适合 AquaZFS 的时候将 ExtFS 作为后备
        \item 进行读写请求分离
    \end{enumerate}
\end{enumerate}


    \section{系统实现}

针对以上设计,我们实现了 AquaFS 文件系统。
AquaFS 文件系统的实现基于 ZenFS,我们在 ZenFS 的基础上进行了修改和优化,使其支持 RAID 等功能。

在初赛阶段,由于人手不足等问题,我们优先实现 AquaFS 文件系统的以下几个部分:

\begin{enumerate}
  \item AquaFS 文件系统的 RAID 实现
  \item AquaFS 的 IO 加速实现
  \item AquaFS 文件系统的智能调参模块实现
  \item AquaFS 文件系统的功能和性能测试
\end{enumerate}

在复赛阶段,我们将继续完善 AquaFS 文件系统其余模块的实现,包括智能数据分类、通用 VFS 文件系统接口等。

\subsection{AquaFS 文件系统的 RAID 实现}

当前 AquaFS 的 RAID 实现主要分为两种:全盘 RAID 和分区 RAID。这两种 RAID 实现的区别在于,全盘 RAID 的 RAID 单位是磁盘,而分区 RAID 的 RAID 单位是 Zone。

\subsubsection{全盘 RAID 的实现}

在传统的 RAID 实现中,基本都是以磁盘为单位进行数据 RAID 逻辑。以磁盘为单位的 RAID 能够充分运用各个磁盘的数据吞吐,并基于 Linux 的块设备等抽象层提供软件上的 RAID 功能。我们也首先实现了全盘 RAID 的功能,能够将多个 ZNS 配置为 RAID 0、RAID 1 模式。

由于 ZenFS 是一个没有一般 POSIX 接口的文件系统,其仅向上服务于 RocksDB,所以我们并不能简单地使用 Linux 上常用的磁盘软件 RAID 来实现 ZNS 的 RAID 功能。更何况,ZNS 并不是 Linux 兼容的块设备,内核中并没有对 ZNS 的 RAID 支持。经过调研和评估,我们认为在 Linux Kernel 内实现对 ZNS 的 RAID 支持并不现实。于是,我们需要在 ZenFS 的基础上实现 RAID 功能。

ZenFS 的数据读写将会经过以下几个层次:

\begin{enumerate}
  \item RocksDB 的 SSTable
  \item RocksDB 的 MemTable
  \item RocksDB 的 WAL
  \item RocksDB 的 FileSystem
  \item ZenFS 的 ZoneFile
  \item ZenFS 的 ZoneExtent
  \item ZenFS 的 Record
  \item ZenFS 的 ZonedBlockDeviceBackend
  \item libzbd 的 ZbdlibBackend
  \item Linux Kernel 相关系统调用
\end{enumerate}

其中,RocksDB 的 SSTable、MemTable、WAL、FileSystem 都是 RocksDB 的内部实现,ZenFS 的 ZoneFile、ZoneExtent、Record、ZonedBlockDeviceBackend 都是 ZenFS 的内部实现,libzbd 的 ZbdlibBackend 是 ZenFS 内对 libzbd 的接口适配,Linux Kernel 相关系统调用是 libzbd 作为用户态程序调用 Linux Kernel 内的设备驱动程序的接口,最终还是会通过 Linux 的系统调用来实现数据的传输。

ZenFS 和 RocksDB 还支持了另一种数据读写方式,通过 ZoneFS 将 ZNS 中的 Zones 以文件映射到文件系统中,然后通过文件系统的接口来读写数据。这种方式的数据读写流程如下:

\begin{enumerate}
  \item RocksDB 的 SSTable
  \item RocksDB 的 MemTable
  \item RocksDB 的 WAL
  \item RocksDB 的 FileSystem
  \item ZenFS 的 ZoneFile
  \item ZenFS 的 ZoneExtent
  \item ZenFS 的 Record
  \item ZenFS 的 ZonedBlockDeviceBackend
  \item ZoneFS 的 ZoneFSBackend
  \item Linux Kernel VFS 接口
\end{enumerate}

虽然这种方式通过 ZoneFS 将 ZNS 中的 Zones 以文件映射到文件系统中,但是 ZoneFS 并不是一个通用的文件系统,它仅仅是一个将 ZNS 中的 Zones 以文件的形式映射到文件系统中的文件系统。ZoneFS 并不支持文件的创建、删除、重命名等操作,文件的读写操作也支持不完全,而且还会引入更多的存储 IO 栈,所以我们并没有选择通过 ZoneFS 来实现 ZNS 的 RAID 功能。

我们选择在 ZenFS 的 ZonedBlockDeviceBackend 层实现 ZNS 的 RAID 功能。ZonedBlockDeviceBackend 层是 ZenFS 中管理数据后端的层,它是 ZenFS 与 libzbd 或 ZoneFS 之间的接口层,负责将 ZenFS 的数据读写请求转换为 libzbd 或 ZoneFS 的数据读写请求。

我们通过继承 ZonedBlockDeviceBackend 来实现 ZenFS 内的数据 RAID 功能。ZonedBlockDeviceBackend 的继承类图如图 \ref{raid-layers} 所示。

\begin{figure}[htbp]
  \centering
  \includegraphics[width=0.85\textwidth]{fig/raid-layers.png}
  \caption{ ZonedBlockDeviceBackend 继承类图 }
  \label{raid-layers}
\end{figure}

AbstractRaidZonedBlockDevice 下的 Raid0ZonedBlockDevice、Raid1ZonedBlockDevice 和 RaidCZonedBlockDevice 即为全盘 RAID 的实现。其中,Raid0ZonedBlockDevice 实现了 RAID 0 的功能,Raid1ZonedBlockDevice 实现了 RAID 1 的功能,RaidCZonedBlockDevice 实现了 RAID C 的功能。

RAID C 是我们自定义的一种简单 RAID 格式,它通过 Zones 的合并来实现简单的数据拼合逻辑,即将多个 Zone 合并为一个 Zone,然后将数据写入到合并后的 Zone 中。RAID C 的实现如图 \ref{raid-c} 所示。

\begin{figure}[htbp]
  \centering
  \includegraphics[width=0.85\textwidth]{fig/raid-c.png}
  \caption{ 全盘 RAID 的数据排布 }
  \label{raid-c}
\end{figure}

此时多个($m$个)设备上实际存在的 Zones 合并为一个大的逻辑 Raid Zone,数据将会按顺序依次写入 Dev Zone $n_{i}$($0 \le i < m$)。此 RAID C 模式存在的意义是,形成一个简单的 Zone 合并逻辑,便于后续开发应用。其中,Dev Zone $n_i$ 分别为来自第 $i$ 个设备的第 $n$ 个 Zone。

RAID 0 和 RAID 1 的实现与传统的 RAID 0 和 RAID 1 的实现类似,RAID 0 将数据分散写入多个设备中,RAID 1 将数据写入多个设备中的一个。RAID 0 和 RAID 1 的实现中的数据排布与 RAID C (图 \ref{raid-c})基本一致。

RAID 0 在写入时,将会将数据分散写入多个设备中。类似于传统块设备的 Block Size,ZNS 也是有最小写入单位的,也是 Block Size。RAID 0 在读写时,可以将读写请求分割为不同的 Block Size 的读写请求,然后对这些请求重新合并排序,再调用底层的读写接口。

以 RAID 0 读为例,其未经 IO 优化的读流程如下所示:

\begin{lstlisting}
int Raid0ZonedBlockDevice::Read(char *buf, int size, uint64_t pos,
                                bool direct) {
#ifndef AQUAFS_RAID_URING
  // split read range as blocks
  int sz_read = 0;
  int r;
  while (size > 0) {
    auto req_size =
        std::min(size, static_cast<int>(GetBlockSize() - pos % GetBlockSize()));
    r = devices_[get_idx_dev(pos)]->Read(buf, req_size, req_pos(pos), direct);
    if (r > 0) {
      size -= r;
      sz_read += r;
      buf += r;
      pos += r;
    } else {
      return r;
    }
  }
  return sz_read;
#else
  // ...
#endif
}
\end{lstlisting}

代码逻辑主要为,每次请求最多读取一个 Block Size 的数据,然后将读取的数据拼接到 buf 中,直到读取完毕。其中需要多次重复计算数据分块的设备位置和块位置,于是这里使用了 get\_idx\_dev 和 req\_pos 函数来快速计算设备位置和块位置。这些函数被实现在 AbstractRaidZonedBlockDevice 层,以便其子类可以快速调用其逻辑。

在上述代码中,我们将读请求分割为不同的 Block Size 的读请求,然后调用底层的读写接口。这样做的好处是,可以将读请求分散到多个设备中,从而提高读性能。不过,上述代码中其实并没有体现 RAID 0 的多设备读优化,还是单线程读取。我们在后文中实现了基于 uring 的多设备并行读优化。

\subsubsection{智能分区 RAID 的实现}

分区 RAID 的实现与全盘 RAID 的实现类似,但是我们在逻辑 Raid Zone 和实际设备 Zone 之间加上了一层基于 Zones 的映射。这些映射是通过 ZenFS 的 Record 写入 MetaZones 内的,将在每次文件系统加载的时候逐步读取加载映射逻辑。

加上这一层映射之后,分区 RAID 的数据排布逻辑可能如图 \ref{raid-a} 所示。

\begin{figure}[htbp]
  \centering
  \includegraphics[width=0.85\textwidth]{fig/raid-a.png}
  \caption{ 分区 RAID 的数据排布 }
  \label{raid-a}
\end{figure}

Zone Raid $n$ 为向上层暴露出的可读写数据 Zone 区域,而其中可以存在多种不同的 RAID 逻辑或映射类型。

在示例图 \ref{raid-a} 中,Dev Zone $x$ 表示来自不同或相同设备的设备上物理存在的 Zone。
如果这个 Raid Zone 被配置为 RAID 1,则 Dev Zone $a$ 和 Dev Zone $e$ 将同时以 RAID 1 数据冗余方式为 Raid Zone $n$ 的前四分之一数据提供服务,其他 Dev Zones 由于没有映射,将回退到 RAID C 逻辑提供数据存储服务。若这个 Raid Zone 被配置为 RAID 0,则 Dev Zone $a$ 和 Dev Zone $e$ 将同时以 RAID 0 数据分散方式为 Raid Zone $n$ 的前四分之一数据提供服务,其他 Dev Zones 由于没有映射,将回退到 RAID C 逻辑提供数据存储服务。

实际代码实现上,除了上述映射逻辑处理,还有许多细节需要考虑。

首先是跨 Zones 读写问题。由于我们添加的映射逻辑,使得我们的数据排布不再是连续的,而是以 Zones 为单位分散的。这就导致了我们的读写请求可能会跨越多个 Zone。这就需要我们在读写时,需要将读写请求分割为不同的 Block Size 的读写请求,然后可能对这些请求重新合并排序,再调用底层的读写接口。

在读或写代码中,通过数据段分割并递归调用自身的方式实现跨 Zones 读写的请求分割:

\begin{lstlisting}
  if (static_cast<decltype(zone_sz_)>(size) > zone_sz_) {
    // may cross raid zone, split read range as zones
    int sz_read = 0;
    int r;
    while (size > 0) {
      auto req_size =
          std::min(size, static_cast<int>(zone_sz_ - pos % zone_sz_));
      r = Read(buf, req_size, pos, direct);
      if (r > 0) {
        buf += r;
        pos += r;
        sz_read += r;
        size -= r;
      } else {
        return r;
      }
    }
    // flush_zone_info();
    return sz_read;
  } else {
    // ...
\end{lstlisting}

其次是数据后端问题。由于我们是基于 ZonedBlockDeviceBackend 来实现的,而 ZonedBlockDeviceBackend 有多种子类,可以是 libzbd、ZoneFS 甚至是原来实现的全盘 RAID。
为了进一步简化 IO 调用栈,我们在实现分区 RAID 时,假定数据后端都是 libzbd 提供的数据读写。这在之后可以进一步优化,以支持更多的数据后端,提升 RAID 逻辑的灵活性,如添加 ZoneFS 支持、添加 SPDK 等 Kernel bypass 方案支持等。

除此以外,还需要考虑对 ZenFS 的兼容性问题。ZenFS 在加载的过程中,会对固定的 MetaZones 进行扫描,通过 Magic Number 查找到 Meta Zones 中的可用的超级块,并选择最新的超级块进行文件系统初始化。为了兼容 ZenFS 的 Meta Data 管理逻辑,我们不能改变 MetaZones 的排布,也不能改变超级块的存储逻辑。因此,我们在实现分区 RAID 时,需要保证 MetaZones 的排布不变,超级块的存储逻辑不变,以及超级块的存储位置不变。所以,我们在实现分区 RAID 时,将 MetaZones 的排布和超级块的存储位置都固定在了第一个设备上,进行映射的连续逻辑预分配,这样就可以保证 ZenFS 的兼容性,使得 ZenFS 在加载分区 RAID 时,可以正常加载。

在创建或读取文件系统时的预分配映射:

\begin{lstlisting}
  // create temporal device map: AQUAFS_META_ZONES in the first device is used
  // as meta zones, and marked as RAID_NONE; others are marked as RAID_C
  for (idx_t idx = 0; idx < AQUAFS_META_ZONES; idx++) {
    for (size_t i = 0; i < nr_dev(); i++)
      allocator.addMapping(idx * nr_dev() + i, 0, idx * nr_dev() + i);
    allocator.setMappingMode(idx, RaidMode::RAID_NONE);
  }
\end{lstlisting}

为了管理分区之间的映射关系,我们通过组合的方式实现了一个分区分配器 ZoneRaidAllocator。其可以管理分区的映射关系,以及分区的 RAID 逻辑。其主要接口如下:

\begin{lstlisting}
  Status addMapping(idx_t logical_raid_zone_sub_idx, idx_t physical_device_idx,
                    idx_t physical_zone_idx);
  void setMappingMode(idx_t logical_raid_zone_idx, RaidModeItem mode);
  void setMappingMode(idx_t logical_raid_zone_idx, RaidMode mode);

  int getFreeDeviceZone(idx_t device);
  int getFreeZoneDevice(idx_t device_zone);
  Status createMapping(idx_t logical_raid_zone_idx);
  Status createMappingTwice(idx_t logical_raid_zone_idx);
  Status createOneMappingAt(idx_t logical_raid_zone_sub_idx, idx_t device,
                            idx_t &zone);
  void setOffline(idx_t device, idx_t zone);
\end{lstlisting}

其可以提供映射关系的查询、增加、删除、修改等功能,以及提供 Raid Zone 所分配的 RAID 逻辑的查询、增加、删除、修改等功能。

同时,它还提供了 setOffline 功能,可以在发现设备故障时,将故障设备的所有分区设置为 Offline 状态,以便后续的故障处理。

\subsection{AquaFS 的 IO 加速实现}

\subsection{AquaFS 文件系统的智能调参模块实现}

智能调参模块方面,实现了基于方差的重要参数选择方案和基于高斯过程回归的参数调整方案。

\begin{figure}[htbp]
  \centering
  \includegraphics[width=0.7\textwidth]{fig/aquaturnner.png}
  \caption{AquaTurnner 智能调参模块}
  \label{aquaturnner}
\end{figure}

如图 \ref{aquaturnner} 所示,AquaTurnner 智能调参模块主要由三部分组成,被调对象,本文中是AquaFS,历史数据存储仓库,以及调参事务组成,调参事务又包括选择参数选择模块和高斯过程回归调整参数模块。

首先,AquaFS需要预热地运行,收集到不同参数配置下的目标指标,本文中使用的是吞吐量指标,将对应的参数配置和目标指标值存入数据仓库中。在初次预热之后,后续不须要预热,除非加入了新的参数指标。

考虑到AquaFS作为RocksDB的插件存在,在本文中使用RocksDB的测试脚本db\_bench来测试系统的吞吐量,采用prometheus作为收集数据的工具。在这个测试中,用收集到的目标指标值和配置参数值,配合历史数据仓库中的数据,来进行参数选择和高斯过程回归。参数选择模块根据方差指标选择最重要的几个参数,高斯过程回归用重要的参数和目标指标值进行拟合以及回归。

首先对于最重要的参数,由方差计算的最大值对应的参数得到:

\begin{equation}
  \label{eq:var}
  \begin{aligned}
    Var(S)=\frac{1}{\lvert S\rvert}\sum_{i=1}^{\lvert S\rvert}(y_i-\mu)^2 \\
    PI(P)=Var(S)-\sum_{i=1}^{N}\frac{S_{P=P_i}}{S}Var(S_{p=p_i})
  \end{aligned}
\end{equation}

这里的 $y_i$ 是样本中的目标值,$\mu$ 是样本目标值均值,$PI$ 系数是基于方差来计算的,即固定某一个参数的值,根据参数的值划分集合,在每个集合中求出集合中目标值的对应方差,再用初始方差 $Var(S)$ 减去这个和,这个 $PI$ 系数越大说明原来的这个参数的影响越大,因为在同一个值的情况下,集合内方差的和很小。

其次由 $CPI$ 指标来选择剩余重要参数:

\begin{equation}
  \label{eq:cpi}
  \begin{aligned}
    CPI(Q|P=p) = Var(S_{P=p})-\sum_{j=1}^{m}\frac{S_{Q=Q_{w_j},P=p}}{S_{P=p}}Var(S_{Q=q_i|P=p})\\
    CPI(Q|P=p) = \max_{1\le i\le n}CPI(Q|P=p_i)
  \end{aligned}
\end{equation}

基于方差的重要参数选择算法如算法 \ref{alg:aquaturnner_select} 所示。

\begin{algorithm}[htb]
  \caption{ AquaTuner参数选择算法 }
  \label{alg:aquaturnner_select}
  \begin{algorithmic}[1]
    \Require
      adjust\_param\_num, db\_bench\_data, data in repository
    \Ensure
      important\_params
    \State important\_params = []
    \State Select the most important param by $PI(param)$, add to important\_params;
    \State For $i$ in adjust\_param\_num $–$ $1$:
    \State \qquad Compute $CPI$ for each param that not in important\_params;
    \State \qquad Select the largest $CPI$’s corresonding params to important params; \\
    \Return important\_params
  \end{algorithmic}
\end{algorithm}

对于连续参数,AquaTuner对于参数指标在参数范围内给出合适的推荐参数值,该合适参数值是在拟合的高斯过程模型曲线上,根据在最优配置参数附近做抖动获得,也即尝试最优配置参数点附近的参数值看目标指标是否有所提升,对于离散参数,AquaTunar尝试匹配最优目标指标值对应的配置参数的离散值。

AquaTuner的运行算法流程如算法 \ref{alg:aquaturnner_trunning} 所示。

\begin{algorithm}[htb]
  \caption{ AquaTuner参数调优算法 }
  \label{alg:aquaturnner_trunning}
  \begin{algorithmic}[2]
    \Require
      adjust\_param\_num
    \Ensure
      recommend\_param
    \State data repository <- warm up system and collect data;
    \State start db\_bench;
    \State db\_bench\_data = collect data from db\_bench;
    \State add db\_bench\_data to data repository;
    \State important\_params = Select\_Param(adjust\_param\_num ,db\_bench\_data, data in repository);
    \State GP\_model = GP\_regression(important\_params, eb\_bench\_data, data in repository);
    \State recommend\_param = [];
    \State For param in  history\_best\_params and important params:
    \State \qquad If param is continuous:
    \State \qquad \qquad Try values near the past value, add to recommend\_param;
    \State \qquad Else if param is discrete:
    \State \qquad \qquad Try values in best params,add to recommend\_param;
    \State Target = GP\_model.predict(recommend\_param);
    \State If Target is better:
    \State \qquad \Return recommend\_param;
    \State \Return history\_best\_param;
  \end{algorithmic}
\end{algorithm}

    \section{系统测试}

\subsection{智能分区 RAID 模块和 IO 加速测试}

由于 ZNS 当前仅有西数公司的设备支持,我们原本预计使用西数的 ZNS SSD 进行测试,即 SN540 512GB。但是由于西数的 SSD 申请过程非常复杂,国内外沟通流程也比较慢,现在仍在处理流程中,我们无法在短时间内获得设备,只能使用仿真环境进行测试。

在几个月前,FAST23 上有一篇关于 ZNS 的论文 nvmevirt\cite{kim_nvmevirt_nodate},它类似 SPDK,在用户态实现了 NVME 的协议,也包括了 ZNS 的协议,可以在 Linux 上模拟 ZNS SSD。但是我们在测试过程中发现,它的 ZNS 模拟实现有问题,无法正常使用,我们在 Github 上提了 issue,但是现在项目还在持续更新中,我们测试中遇到的问题只解决了一些,但是又有更多新的问题出现。我们尝试去解决这些问题,但是由于时间有限,我们最终无法在初赛前解决这些问题,只能放弃在初赛中使用这个模拟器。

以下 RAID IO 测试均使用了基于数据的延迟计算仿真,即计算 SSD 的访问延迟和数据传输延迟,其数据存储后端是内存。由于内存的访问延迟远小于 SSD,因此可以忽略内存的访问延迟,只计算 SSD 的访问延迟和数据传输延迟。

测试环境为:Intel(R) Core(TM) i7-12700 CPU,内存 40GB 3200MHz;系统为 Archlinux,Linux 内核版本 6.3.5-arhc1-1。由于 libzbd、ZoneFS、nvmevirt 等都需要比较高的内核版本,于是这里直接一步到位地使用了最新的 Linux Kernel mainline。

\subsubsection{AquaZFS 数据完整性测试}

由于 AquaZFS 是作为一个插件运行在 RocksDB 的文件系统层上的,所以可以使用 RocksDB 来验证其数据完整性。我们使用 RocksDB 的 db\_bench 工具进行测试,测试结果如表 \ref{test-data} 所示。

\begin{table}[htbp]
  \centering
  \caption{数据完整性测试}
  \label{test-data}
  \begin{tabular}{cccc}
    \hline
    \textbf{测试项} & \textbf{设备数} & \textbf{测试参数} & \textbf{测试结果} \\
    \hline
    全盘 RAID C & 2 & \verb|aquafs://raidc:dev:nullb0,dev:nullb1| & 通过 \\
    全盘 RAID 0 & 2 & \verb|aquafs://raid0:dev:nullb0,dev:nullb1| & 通过 \\
    全盘 RAID 1 & 2 & \verb|aquafs://raid1:dev:nullb0,dev:nullb1| & 通过 \\
    全盘 RAID 5 & 2 & \verb|aquafs://raid5:dev:nullb0,dev:nullb1| & 通过 \\
    分区 RAID A & 2 & \verb|aquafs://raida:dev:nullb0,dev:nullb1| & 通过 \\
    \multirow{2}{*}{分区 RAID A} & \multirow{2}{*}{4} & \verb|aquafs://raida:dev:nullb0,dev:nullb1,| & \multirow{2}{*}{通过} \\
    & & \verb|dev:nullb2,dev:nullb3| & \\
    \hline
  \end{tabular}
\end{table}

db\_bench 除了表 \ref{test-data} 中的,还需要添加如下参数:

\begin{lstlisting}
./plugin/aquafs/aquafs
  mkfs # 创建文件系统
  # --raids= 此选项用于指定数据后端,这里指定使用 AquaFS 的 RAID 功能
  --aux_path=/tmp/aquafs # 指定 AquaFS 的辅助数据存储路径,如锁文件和日志缓存文件
  --force # 清除原有的文件系统数据

./db_bench 
  # --fs_uri= 此选项用于指定数据后端,可以是文件系统路径,也可以是 AquaFS 的相关配置
  --benchmarks=fillrandom # 指定测试项,使用随机写并校验
  --use_direct_io_for_flush_and_compaction # 使用 Direct IO 加速
  --use_stderr_info_logger # 在标准错误输出日志
\end{lstlisting}

由于输出较多,这里仅展示一部分输出,如图 \ref{check-data1}、\ref{check-data2}、\ref{check-data3} 所示。

\begin{figure}[htbp]
  \centering
  \includegraphics[width=0.85\textwidth]{fig/raid-data-check1}
  \caption{ 数据完整性测试1 }
  \label{check-data1}
\end{figure}

\begin{figure}[htbp]
  \centering
  \includegraphics[width=0.85\textwidth]{fig/raid-data-check2}
  \caption{ 数据完整性测试2 }
  \label{check-data2}
\end{figure}

\begin{figure}[htbp]
  \centering
  \includegraphics[width=0.85\textwidth]{fig/raid-data-check3}
  \caption{ 数据完整性测试3 }
  \label{check-data3}
\end{figure}

\subsubsection{使用 RAID 0 加速文件读写}

由于 ZenFS 并不支持 POSIX 规范的文件系统访问接口,所以无法将其挂载到 Linux 的 VFS 上,因此无法使用 Linux 的文件系统测试工具来测试其性能。这里使用的测试逻辑,与 RocksDB 使用 AquaFS 文件系统插件类似,都是通过写程序调用 AquaFS 的 API 来测试性能。

使用如下逻辑测试文件读写性能,并比较 RAID 0 的性能加速:

\begin{enumerate}
  \item 生成测试参数,包括文件大小、设备数量、RAID逻辑、随机种子等
  \item 创建文件系统,设置RAID逻辑等
  \item 在内存中生成一个指定大小的随机内容文件
  \item 重复进行多次:
  \begin{enumerate}
    \item 通过 \verb|aquafs restore| 将这个文件写入 AquaFS 文件系统
    \item 通过 \verb|aquafs dump| 将文件从 AquaFS 文件系统读出
    \item 通过 \verb|std::hash| 和 \verb|md5sum| 比较读出的文件和原文件是否一致
  \end{enumerate}
  \item 计算平均耗时,得到读写平均时间
\end{enumerate}

对多组数据进行测试,结果输出如图 \ref{raid0-speedup} 所示:

\begin{figure}[htbp]
  \centering
  \includegraphics[width=0.85\textwidth]{fig/raid0-speedup}
  \caption{ RAID 0 加速测试 }
  \label{raid0-speedup}
\end{figure}

其得到的数据如表 \ref{raid0-speedup-table} 所示:

\begin{table}[htbp]
  \centering
  \caption{RAID 0 加速测试,5 次运行取平均}
  \label{raid0-speedup-table}
  \begin{tabular}{cccccc}
    \hline
    \textbf{设备数量} & \textbf{文件大小(MiB)} & \textbf{平均读时间(ms)} & \textbf{平均写时间(ms)} & \textbf{说明} \\
    \hline
    1 & 128 & 209 & 114 & 单盘分区 RAID \\
    4 & 128 & 51 & 31 & 多盘全盘 RAID 0 \\
    4 & 128 & 198 & 106 &  多盘分区 RAID \\
    1 & 128 & 1467 & 1431 &  ZenFS 默认 \\
    \hline
  \end{tabular}
\end{table}

可以看到,在单盘分区 RAID 的情况下,读写性能都有至少 N 倍的性能提升,而在多盘全盘 RAID 0 的情况下,读写性能有更大提升。

相比与 ZenFS 的默认读写,单盘分区读写由于使用 io\_uring、协程、重新构造读写请求等方式,性能有成倍的提升。具体提升幅度被能同时读写的 Zone 数量限制;而原版 ZenFS 只能单线程进行数据请求,因此性能较低。

相比与单设备的分区 RAID,多盘分区 RAID 有更大的性能提升。多盘意味着系统并行度更大,因此获得了更高的瞬时吞吐量。

而多盘全盘 RAID 0,由于采用的读写加速方式与分区 RAID 一致,其逻辑更加精简,因此性能更高。

\subsubsection{文件系统数据恢复测试}

测试逻辑:

\begin{enumerate}
  \item 生成随机大文件
  \item 使用 \verb|aquafs restore| 将文件写入 AquaFS 文件系统
  \item 在软件层面触发模拟硬件故障,使得一个 Dev Zone 下线
  \item 使用 \verb|aquafs dump| 将文件从 AquaFS 文件系统读出,观察是否能够正常读出
  \item 校验读出的文件和原文件是否一致
\end{enumerate}

测试流程图如下所示:
\begin{figure}[htbp]
  \centering
  \includegraphics[width=0.85\textwidth]{fig/recovery_logic}
  \caption{ 测试流程图 }
  \label{recovery_logic}
\end{figure}

使用 128 MiB 文件大小进行测试,测试过程如图 \ref{test-recovery} 所示。
在测试中,AquaFS 成功地检测到了 Zone 的故障,并且成功地从其他 Zone 中恢复了数据,维护了文件系统的数据完整性。

\begin{figure}[htbp]
  \centering
  \includegraphics[width=0.85\textwidth]{fig/test-recovery}
  \caption{ 数据恢复测试 }
  \label{test-recovery}
\end{figure}

其中,测试的RAID0和RAID5的恢复得到了正确结果:
\begin{figure}[htbp]
  \centering
  \includegraphics[width=0.85\textwidth]{fig/test_recovery_1}
  \caption{ RAID0 RAID5数据恢复测试 }
  \label{test-recovery_1}
\end{figure}

\subsubsection{文件系统性能测试}

由于可以使用内存作为存储后端,我们可以很便利地分析 AquaFS 的性能瓶颈,从而快速优化其性能。

我们使用 Linux \verb|perf| 工具进行性能分析,其中部分分析过程如图 \ref{test-perf} 所示。

\begin{figure}[htbp]
  \centering
  \includegraphics[width=0.85\textwidth]{fig/test-perf}
  \caption{ perf 性能分析 }
  \label{test-perf}
\end{figure}

可以看到,AquaFS 的性能瓶颈主要在于:

\begin{enumerate}
  \item \verb|entry_SYSCALL_64|:当前仍然使用内核态的 \verb|open|、\verb|read|、\verb|write| 等系统调用,导致系统频繁地从用户态切换到内核态,导致性能瓶颈。
  \item \verb|io_submit_sqes|:正在使用 io\_uring 的软件层面实现,可能使用方法还有可以优化的地方,造成 io\_uring 中一部分数据的锁争用。
\end{enumerate}

\subsubsection{ExtFS 功能和性能测试}

为了测试我们 ExtFS 的功能正确性,我们使用了本校操作系统实验课程中的测试脚本,对 ExtFS 进行了完整的功能测试,测试结果如下所示:

\begin{lstlisting}[language=bash]
  $ ./test.sh
  开始mount, mkdir, touch, ls, read&write, cp, umount测试测试脚本工程根目录: 
    /home/chiro/os/fuse-ext2/fs/rfs/rfs/tests
  测试用例: /home/chiro/os/fuse-ext2/fs/rfs/rfs/tests/stages/mount.sh
  测试用例: /home/chiro/os/fuse-ext2/fs/rfs/rfs/tests/stages/mkdir.sh
  测试用例: /home/chiro/os/fuse-ext2/fs/rfs/rfs/tests/stages/touch.sh
  测试用例: /home/chiro/os/fuse-ext2/fs/rfs/rfs/tests/stages/ls.sh
  测试用例: /home/chiro/os/fuse-ext2/fs/rfs/rfs/tests/stages/remount.sh
  测试用例: /home/chiro/os/fuse-ext2/fs/rfs/rfs/tests/stages/rw.sh
  测试用例: /home/chiro/os/fuse-ext2/fs/rfs/rfs/tests/stages/cp.sh
  =================================    Finished dev [unoptimized + debuginfo] target(s) in 0.05s
       Running `/home/chiro/os/fuse-ext2/fs/rfs/rfs/target/debug/rfs --device=/home/chiro/ddriver 
          -q /home/chiro/os/fuse-ext2/fs/rfs/rfs/tests/mnt`
  pass: case 1 - mount
  ======================pass: case 2.1 - mkdir /home/chiro/os/fuse-ext2/fs/rfs/rfs/tests/mnt/dir0
  pass: case 2.2 - mkdir /home/chiro/os/fuse-ext2/fs/rfs/rfs/tests/mnt/dir0/dir0
  pass: case 2.3 - mkdir /home/chiro/os/fuse-ext2/fs/rfs/rfs/tests/mnt/dir0/dir0/dir0
  pass: case 2.4 - mkdir /home/chiro/os/fuse-ext2/fs/rfs/rfs/tests/mnt/dir1
  =======================pass: case 3.1 - touch /home/chiro/os/fuse-ext2/fs/rfs/rfs/tests/mnt/file0
  pass: case 3.2 - touch /home/chiro/os/fuse-ext2/fs/rfs/rfs/tests/mnt/file1
  pass: case 3.3 - touch /home/chiro/os/fuse-ext2/fs/rfs/rfs/tests/mnt/dir0/file1
  pass: case 3.4 - touch /home/chiro/os/fuse-ext2/fs/rfs/rfs/tests/mnt/dir0/file2
  pass: case 3.5 - touch /home/chiro/os/fuse-ext2/fs/rfs/rfs/tests/mnt/dir1/file3
  ======================pass: case 4.1 - ls /home/chiro/os/fuse-ext2/fs/rfs/rfs/tests/mnt/
  pass: case 4.2 - ls /home/chiro/os/fuse-ext2/fs/rfs/rfs/tests/mnt/dir0
  pass: case 4.3 - ls /home/chiro/os/fuse-ext2/fs/rfs/rfs/tests/mnt/dir0/dir1
  pass: case 4.4 - ls /home/chiro/os/fuse-ext2/fs/rfs/rfs/tests/mnt/dir0/dir1/dir2
  ==============================    Finished dev [unoptimized + debuginfo] target(s) in 0.05s
       Running `/home/chiro/os/fuse-ext2/fs/rfs/rfs/target/debug/rfs --device=/home/chiro/ddriver 
        -q /home/chiro/os/fuse-ext2/fs/rfs/rfs/tests/mnt`
  pass: case 5.1 - umount /home/chiro/os/fuse-ext2/fs/rfs/rfs/tests/mnt
  pass: case 5.2 - check bitmap
  ================================    Finished dev [unoptimized + debuginfo] target(s) in 0.06s
       Running `/home/chiro/os/fuse-ext2/fs/rfs/rfs/target/debug/rfs --device=/home/chiro/ddriver 
        -q /home/chiro/os/fuse-ext2/fs/rfs/rfs/tests/mnt`
  pass: case 6.1 - write /home/chiro/os/fuse-ext2/fs/rfs/rfs/tests/mnt/file0
  pass: case 6.2 - read /home/chiro/os/fuse-ext2/fs/rfs/rfs/tests/mnt/file0
  =======pass: case 7.1 - prepare content of /home/chiro/os/fuse-ext2/fs/rfs/rfs/tests/mnt/file9
  pass: case 7.2 - copy /home/chiro/os/fuse-ext2/fs/rfs/rfs/tests/mnt/file9 to 
      /home/chiro/os/fuse-ext2/fs/rfs/rfs/tests/mnt/file10
  =============================================== 
  Score: 34/34
  pass: 通过所有测试 (34/34)
\end{lstlisting}

接着使用 \verb|dd| 工具,在本机的普通 1TiB SSD 上进行了简单的单文件读写测试:

\begin{lstlisting}[language=bash]
$ dd if=/dev/random of=mnt/random bs=1MiB count=64
输入了 64+0 块记录输出了 64+0 块记录67108864 字节 (67 MB, 64 MiB) 已复制,5.88972 s,11.4 MB/s
$ dd of=/dev/null if=mnt/random bs=1MiB count=64
输入了 64+0 块记录输出了 64+0 块记录67108864 字节 (67 MB, 64 MiB) 已复制,0.635838 s,106 MB/s
\end{lstlisting}

使用实验中的脚本进行读写性能测试,其测试逻辑为对文件中的某一固定的块反复读写:

\begin{lstlisting}
// 关闭模拟磁盘延迟,并且打开了缓存
Test loop: 1000000, Cache Blks: 512
  Finished release [optimized] target(s) in 0.05s
   Running `target/release/rfs --format -q -c --cache_size 512 /home/chiro/mnt`
Time: 30815.510034561157ms BW: 253.52492920733374MB/s
// 关闭模拟磁盘延迟,并且关闭了缓存
Test loop: 100000, Cache Blks: 0
  Finished release [optimized] target(s) in 0.05s
   Running `target/release/rfs --format -q /home/chiro/mnt`
Time: 8691.662073135376ms BW: 89.88499477156695MB/s
// 打开了模拟磁盘延迟,并且关闭了缓存
Test loop: 1000000, Cache Blks: 512
  Finished release [optimized] target(s) in 0.06s
   Running `target/release/rfs --format -q --latency -c --cache_size 512 /home/chiro/mnt`
Time: 30927.71863937378ms BW: 252.6051174707074MB/s
// 打开了模拟磁盘延迟,并且关闭了缓存
Test loop: 100, Cache Blks: 0
  Finished release [optimized] target(s) in 0.05s
   Running `target/release/rfs --format -q --latency /home/chiro/mnt`
Time: 9035.56513786316ms BW: 0.08646387780728894MB/s
\end{lstlisting}

可以看到,关闭缓存后,读写性能相比有缓存的大幅下降;而开启缓存后,读写性能相比无缓存的大幅提升。ExtFS 上的缓存工作正常且非常有效。在关闭了手动管理的缓存后,由于 Linux 本身为块设备提供了缓存,因此在不模拟磁盘延迟时关闭了手动管理的缓存,读写性能并没有下降太多。

由于 ExtFS 完成度还不高,所以 ExtFS 暂时还不是本项目的重点,后续会继续完善 ExtFS 的功能。

\subsection{AquaTurnner 智能调参模块测试}

首先进行warm\_up得到一组参数和目标指标值的csv文件如图 \ref{test-turnner1} 所示,现在由于只调整AquaFS参数,参数空间较小,主要包括垃圾回收的gc相关参数,块大小以及zone的大小参数等,后续考虑在融合文件系统后加入inode相关参数。

\begin{figure}[htbp]
  \centering
  \includegraphics[width=0.85\textwidth]{fig/turnner1}
  \caption{ 调参测试数据 }
  \label{test-turnner1}
\end{figure}

设定参数进行AquaTuner的参数推荐流程,离散参数直接指定:

\begin{lstlisting}[language=Python]
SECT_SIZE_PARAM = 128
ZONE_SIZE_PARAM = 64
\end{lstlisting}

连续参数可以直接通过脚本defconfig获取。

接下来利用db\_bench进行参数跑分以及收集参数和目标值:

\begin{lstlisting}[language=Python]
  pre_throughput = 0
  now_throughput = 0
  for _ in range(1):
      sect_size = SECT_SIZE_PARAM
      zone_size = ZONE_SIZE_PARAM
      total_throughput = []
      for i in range(2):
          create_null_blk(sect_size, zone_size, 0, 64)
          os.system(CREATE_TMP_FILE)
          throughput_list = execute_adjust_param(2, sect_size, zone_size)
          total_throughput = total_throughput + throughput_list
          print("throughput list : {}".format(total_throughput))
          remove_null_blk()
      print("sect_size:{}".format(sect_size))
      print("zone_size:{}".format(zone_size))
      pre_throughput = now_throughput
      now_throughput = np.average(total_throughput)
\end{lstlisting}

代码的整体流程是,首先创建zone块,再在zone块上创建文件,得到跑分的吞吐量以及推荐参数,再将块zone块删除。

创建文件系统如下图所示,创建AquaFS的数据模块nullb0和作为log等文件存储的模块/tmp/aquafs。

\begin{lstlisting}[language=bash]
  CREATE_TMP_FILE = "mkdir -p /tmp/aquafs ;\
  sudo ../build/plugin/aquafs/aquafs mkfs --zbd nullb0 --aux-path /tmp/aquafs"
\end{lstlisting}

之后用db\_bench进行跑分,并进行智能调参模块的逻辑,按照指定推荐的离散参数跑一遍智能调参模块的结果如图 \ref{test-turnner2} 所示。

\begin{figure}[htbp]
  \centering
  \includegraphics[width=0.85\textwidth]{fig/turnner2}
  \caption{ 调参测试过程 }
  \label{test-turnner2}
\end{figure}

在上图中推荐了两个参数,最重要的参数是sect\_size,其次是zone\_size,这符合现阶段的测试预期,因为我们的垃圾回收也就是gc还没有进行触发,,所以能够更改的参数现阶段只有块大小和zone\_size,而调参模块也根据现阶段的数据仓库中最优的参数进行了配置推荐。记录这里的平均吞吐量是13672167.25。

采用接下来的配置再进行测试:

\begin{lstlisting}[language=Python]
SECT_SIZE_PARAM = 512
ZONE_SIZE_PARAM = 32
\end{lstlisting}

最终结果如图 \ref{test-turnner3} 所示。

\begin{figure}[htbp]
  \centering
  \includegraphics[width=0.85\textwidth]{fig/turnner3}
  \caption{ 调参测试结果 }
  \label{test-turnner3}
\end{figure}

平均吞吐量是16935800.25。相比于上一次的配置参数,吞吐量提升了23.87\%。

将不同轮次的调参结果进行对比如图 \ref{test-turnner} 所示,系统能够根据系统运行数据调整文件系统参数,从而逐步提高数据吞吐量。

\begin{figure}[htbp]
  \centering
  \includegraphics[width=0.85\textwidth]{fig/test-turnner}
  \caption{ 调参测试结果对比 }
  \label{test-turnner}
\end{figure}

后续还将尝试测试触发gc来测试gc的参数,以及尝试融合文件系统后对整体的inode以及其他参数进行自动调整。


    \newpage
    \section{总结与展望}

\subsection{工作总结}

我们提出了 AquaFS,一个以 ZenFS 为原型的,基于 Zoned Namespace SSD 的高性能、高灵活性、高效,而且更加智能的适用于 Flash 存储介质的文件系统。

AquaFS 采用模块化设计,不同模块各司其职,分别负责不同的功能,使得 AquaFS 的功能更加清晰,易于维护和扩展。

在当前初赛阶段,我们实现了 AquaFS 的一些核心功能,包括:

\begin{itemize}
  \item 
\end{itemize}
    %% APPENDIX
    \appendix
    \section{图表索引}
\subsection{图目录}
\subsection{表目录}
\newpage
\section{参考文献}
\printbibliography[heading=bibintoc]
\end{document}