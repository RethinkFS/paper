\section*{摘要}
\addcontentsline{toc}{section}{摘要}

ZenFS 由于其简单且与硬件 Zoned Storage SSD 紧密结合等特点,实现了零硬件预留空间、零硬件垃圾回收开销等高负载场景下的高性能特性,但是其仅适合 ZNS 设备、需要软件特殊适配等特点也限制了其应用场景灵活性。

AquaFS 是一个以 ZenFS 为原型的适用于 Zoned Storage SSD 的文件系统,将原来的 ZenFS 模块化,扩展其应用场景,添加 RAID 来平衡数据安全和写放大,同时添加调参模块以提高文件系统的智能性和性能。

在未来,我们将继续实现 AquaFS 的更多功能,如用户态 NVME 驱动加速、更完善的数据恢复、多文件系统融合调参等,进一步扩展应用场景并提升其智能化水平。

初赛阶段进度情况如表 \ref{progress} 所示。

\begin{table}[htbp]
  \centering
  \caption{初赛进度情况}
  \label{progress}
  % \begin{table}[htbp]
%   \centering
%   \caption{初赛进度情况}
%   \label{progress}
  \begin{tabular}{|l|l|p{8cm}|}
    \hline
    \textbf{改进内容}              & \textbf{完成进度} & \textbf{完成情况} \\
    \hline
    智能调参模块                   & 进度约70\%      & \checkmark 基于方差的重要参数选择方案 \\
                                  &                & \checkmark 基于高斯回归的调参方案 \\
                                  &                & $\square$ 将对垃圾回收参数进行进一步测试 \\
                                  &                & $\square$ 将对融合文件系统后的更多参数进行调优测试 \\
    \hline
    基于 Zone 的                  & 整体进度约 80\% & \checkmark 全盘 RAID 模式 RAID0、RAID1、RAID-C \\
    智能动态分区 RAID              &                & \checkmark 分区 RAID 模式 Zone 映射和 RAID 逻辑分配 \\
                                  &                & \checkmark 数据完整性检测和恢复 \\
    \hline
    异步 IO 优化                  & 进度约 60\%     & \checkmark 完成了 \verb|io_uring| 异步读写优化 \\
                                  &                & $\square$ 将进一步研究基于 spdk、xnvme 等的用户态 nvme 驱动 \\
    \hline
    融合通用文件系统               & 进度约 50\%     & \checkmark 基于 FUSE 和 Rust 完成一个 Ext2 兼容文件系统 \\
                                  &                & $\square$ 将进一步研究与 ZenFS 的结合方式 \\
                                  &                & $\square$ 将进一步研究智能数据请求分类方法 \\
    \hline
  \end{tabular}
% \end{table}
\end{table}

使用 RocksDB 随机读写测试来测试 RAID 模块实现的数据正确性,结果如表 \ref{test-data} 所示。在全盘、分区 RAID 情况下进行 RocksDB 数据库读写并且测试通过,RAID 实现的正确性可以得到验证。

使用 \verb|io_uring| 等优化 IO 后,进行单线程读写测试,结果如表 \ref{raid0-speedup-table} 所示。经过异步 IO 单线程读写优化,在不同 RAID 配置下能够充分利用硬件并行性从而提高整体带宽利用率。

对智能调参模块进行测试,由图 \ref{test-turnner},智能调参模块在 RocksDB 随机写入测试中,能够根据系统运行数据调整文件系统参数,从而逐步提高数据吞吐量。