\label{overview}
\section{概述}

AquaFS 是一个以 ZenFS 为原型的适用于 Zoned Storage SSD 的智能化的文件系统,
旨在为闪存设备提供更好的性能、更长的寿命和更适应实际应用场景的运行参数。
本文档将从多个角度介绍 AquaFS 的设计与实现。

AquaFS 与传统 Ext4 等文件系统相比,有以下特点:

\begin{itemize}
    \item 与 Ext4 等原地更新的文件系统不同,AquaFS 采用了日志结构,以适应 Flash 结构特性,减少写放大。
    \item 与传统软件文件系统不同,AquaFS 与硬件结合共同设计,降低硬件成本的同时提升性能。
    \item 与普通的单一文件系统不同,AquaFS 采用了多系统模块结合的方法,以提升系统的灵活性。
    \item 设计了专门的参数调整逻辑,以动态适应不同的工作负载。
    \item 设计了智能的读写逻辑,不同的工作负载能够由不同的模块处理或共同处理,以提升系统的性能。
    \item 优化垃圾回收逻辑,硬件上取消垃圾回收,软件上可调整预留空间与垃圾回收频率,以均衡调整性能与寿命。
\end{itemize}

而与 ZenFS 相比,AquaFS 有以下特点:

\begin{itemize}
    \item \textbf{适用场景更广泛}:ZenFS 仅适用于软件特殊适配后的 ZNS 设备,而 AquaFS 可以通过融合文件系统和通用文件接口等方式适用于更多的软硬件设备。
    \item \textbf{更加灵活}:ZenFS 由于其简单而参数较少,且都由上层软件适配调整,而 AquaFS 可以通过调整参数、融合文件系统等方式提升其场景适用性,为没有软件适配的应用场景提供支持。
    \item \textbf{更加智能}:ZenFS 许多逻辑都是固定的,而 AquaFS 通过智能调参、智能分类读写等方式提升其智能化水平。此外,AquaFS 还可以通过文件系统级冗余、文件系统读写分离、inode 缓存等方式提升其智能化水平。
    \item \textbf{更加安全}:ZenFS 在写入记录层面进行校验,但是对更常见的整块数据损坏无法有效应对,而 AquaFS 通过 RAID 等方式提升了针对 Zones 的数据安全性。
    \item \textbf{更加高效}:ZenFS 仅有 Direct IO 模式,而 AquaFS 支持 \verb|io_uring| 等更高效的异步 IO 模式。
\end{itemize}

后文将结合背景调研、需求分析、系统设计、系统实现、系统测试等方面,对 AquaFS 的各种设计特性进行详细介绍。
